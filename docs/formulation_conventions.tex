\documentclass[onecolumn]{article}
\usepackage{amsmath}
\begin{document}
\title{01 - Conventions}
\author{}
\date{}
\maketitle
\section{Conventions} % (fold)
Following are applicable to calculate the chemical potential of a solution, and for expressing its composition.
\label{sec:conventions}
% section section_name (end)
\subsection{Composition}
\label{subsec:composition}
\begin{table}[h]
\begin{tabular}{|lll|}
\hline
		      & expression                                                                                    & [=] units                     \\
\hline
mole number   & $n_i = \frac{w_i}{M_i}$                                                                                   & mol i                         \\
mole fraction & $x_i = \frac{n_i}{\sum_j{n_j}} =  \frac{n_i}{n_0 + \sum_{j \neq 0}{n_j}}$                         & [adim.]                       \\
molal  conc.  & $m_i = \frac{n_i}{n_0 M_0} = \frac{n_i}{n_0 M_0 / \left(1000 \frac{g}{kg} \right)}$               & $\frac{mol_i} {kg_{solvent}}$ \\
molar  conc.  & $c_i = \frac{n_i}{V_r} = \frac{n_i}{w/\rho} = \frac{n_i \rho}{n_0 M_0 + \sum_{j\neq 0}{n_j M_j}}$ & $\frac{mol_i}{L}$\\ 
              & $= \frac{m_i \rho}{1 + \sum_{j\neq 0}{m_j M_j}}$     &       \\
density 	  & $\rho_i = w_i/V_r = n_i M_i/V_r = c_i M_i$			 &	$\frac{g_i}{mL}$ \\
mass fraction & $x_{w,i} = \frac{w_i}{\sum_j{w_j}} =  \frac{n_i M_i}{n_0 M_0 + \sum_{j \neq 0}{n_j M_j}}$                         & [adim.]                       \\
			  & $= \frac{c_i M_i}{\rho} = \frac{\rho_i}{\rho}$  &  \\
\hline
\end{tabular}
\caption{Expressions of composition. Subscript 0 applied to the solvent, other subscripts applied to solutes.}
\label{table:composition_expr}
\end{table}
Useful associations from these expressions are: \\
\[
\begin{aligned}
\frac{m_i}{x_i} & = \frac{n_i/(n_0 M_0) \times (n_0 + \sum_{j\neq0}{n_j}) }{n_i} = \frac{1}{M_0} + \sum_{j\neq0}{m_j} = \frac{1}{M_0} \times (1 + \sum_{j\neq0}{m_jM_0}) \\
\frac{c_i}{x_i} & = \frac{n_i \rho \times (n_0 + \sum_{j\neq0}{n_j}) }{n_i \times (n_0 M_0 + \sum_{j\neq0}{n_j M_j})} = \frac{ n_0 + \sum_{j\neq0}{n_j} }{ n_0 M_0 + \sum_{j\neq0}{n_j M_j}} \rho = \frac{ \frac{1}{M_0} + \sum_{j\neq0}{m_j} }{ 1 + \sum_{j\neq0}{m_j M_j}} \rho \\
\frac{\rho}{c_0 M_0} & = \frac{\sum_{i}{\rho_i}}{c_0 M_0} = \frac{c_0 M_0 + \sum_{j \neq 0}{c_j M_j}}{c_0 M_0} = 1 + \frac{\sum_{j\neq0}{n_j M_j/V_r}}{n_0 M_0/V_r} = 1+\sum_{j\neq0}{m_j M_j}\\
\end{aligned}
\]
Combining the last two of these expressions, the following is obtained:
\[
\begin{aligned}
\frac{c_i}{x_i} & = \frac{ \frac{1}{M_0} + \sum_{j\neq0}{m_j} }{ 1 + \frac{\rho}{c_0 M_0} -1} \rho = \frac{\rho_0}{M_0} \times (1 + \sum_{j\neq0}{m_jM_0}) \\
				& = \rho_0 \frac{m_i}{x_i}
\end{aligned}
\]
\subsection{Chemical potential \cite{Denbigh1968} }
\label{subsec:chemical_potential}
Suitable conventions are applied to different solutions: Convention I is useful for solutions of miscible liquids at T,P (e.g. ethanol/water). Conventions II, III, and IV are useful for solutions of components that are gases or solids at T,P (e.g. oxygen or sugar in water). \\
Convention I: Each component approaches ideality as its mole fraction approaches unity.
\begin{equation}
\textnormal{component (i): } (\gamma_i \rightarrow 1) as  (x_i \rightarrow 1)\\
\end{equation}
Convention II: Ideal solvent as it approaches unit mole fraction; ideal solutes at \textit{infinite dilution}, expressed in mole fraction.\\
\begin{equation}
\begin{aligned}
\textnormal{solvent (0): } & (\gamma_0 \rightarrow 1) as  (x_0 \rightarrow 1)\\
\textnormal{solute (i): } & (\gamma_i \rightarrow 1) as (x_i \rightarrow 0)
\end{aligned}
\end{equation}
Convention III: Ideal solvent as it approaches unit mole fraction; ideal solutes at \textit{infinite dilution}, expressed in molality. \\
\begin{equation}
\begin{aligned}
\textnormal{solvent (0): } & (\gamma_0 \rightarrow 1) as  (x_0 \rightarrow 1)\\
\textnormal{solute (i): } & (\gamma_i \rightarrow 1) as (m_i \rightarrow 0)
\end{aligned}
\end{equation}
Convention IV: Ideal solutes at \textit{infinite dilution}, expressed in molarity. \\
\begin{equation}
\begin{aligned}
\textnormal{solute (i): } & (\gamma_i \rightarrow 1) as (c_i \rightarrow 0)
\end{aligned}
\end{equation}
Chemical potential expressed under each convention, with $m_i^{\circ}$ and $c_i^{\circ}$selected as reference solute states at unit molality, and molarity respectively.
\begin{equation}
\begin{aligned}
\mu_i = & \mu_i^{*} + RTln(\gamma_i^{I} x_i) \\
	= 	& \mu_i^{*} + RTln(\gamma_i^{II} x_i) \\
	=	& \mu_i^{\fbox{}} + RTln(\gamma_i^{III} m_i/m_i^{\circ}) \\
	= 	& \mu_i^{\diamond} + RTln(\gamma_i^{IV} c_i/c_i^{\circ})
\end{aligned}
\end{equation}
Notes on conversion between conventions: 
\begin{itemize}
\item Values of activity coefficients $\gamma_i^{II}$ and $\gamma_i^{III}$ vary, but chemical potential does not vary due to convention selection.
\item Reference states chemical potentials under each convention $\mu_i^{*}$ and $\mu_i^{\fbox{}}$  are functions of temperature and pressure exclusively. Their difference calculated at infinite dilution does not change at other compositions:
\end{itemize}
\[
\mu_i^{*} - \mu_i^{\fbox{}}  =  func(T,P) =  \lim_{x_0 \to 1} \left( \mu_i^{*} - \mu_i^{\fbox{}}\right) 
\]
\[
RTln\left[ \frac{\gamma_i^{III} m_i/m_i^{\circ}}{\gamma_i^{II} x_i}\right] = \lim_{x_0 \to 1} RTln\left[ \frac{\gamma_i^{III} m_i/m_i^{\circ}}{\gamma_i^{II} x_i}\right] 
\]
Therefore, referring back to conventions II and III, $\gamma_i \rightarrow 1$ at \textit{infinite dilution}:
\[
\begin{aligned}
\frac{\gamma_i^{III} m_i/m_i^{\circ}}{\gamma_i^{II} x_i} & = \lim_{x_0 \to 1} \frac{\gamma_i^{III} m_i/m_i^{\circ}}{\gamma_i^{II} x_i} = \frac{1}{m_i^0} \times \lim_{x_0 \to 1} \frac{m_i}{x_i} \\
& = \frac{1}{m_i^0} \times \lim_{x_0 \to 1} \frac{n_0+\sum_{j \neq 0}n_j}{n_0 M_0} = \frac{1}{m_i^0 M_0} \times \lim_{x_0 \to 1} \frac{1}{x_0} \\
& = \frac{1}{m_i^\circ M_0}
\end{aligned}
\]
This leads to the following relationship between activity coefficients calculated by each convention:
\begin{equation}
\label{eq:act_coef_m_to_x}
\gamma_i^{II} = \gamma_i^{III} \times \frac{m_i M_0}{x_i}
\end{equation}
A standard form of equation \ref{eq:act_coef_m_to_x} is obtained by assuming complete dissociation of a solute $A$ with molality $m$ into $\nu$ components, so that for all solutes the sum of molalities can be expressed as $\sum_{i}m_i = \nu m $:
\[
A \rightleftharpoons A_1 + A_2 + A_2 + ... + A_{\nu}
\]
Combined with Table \ref{table:composition_expr}, the expression as presented in \cite{Hamer1968} is obtained:
\[
\begin{aligned}
\gamma_i^{II} & =  \gamma_i^{III} \times M_0 \frac{m_i}{x_i} = \gamma_i^{III} \times (1+\sum_{j\neq0}{m_j M_0}) = \gamma_i^{III} \times \left(1+\frac{\nu m M_0}{1000 \frac{g}{kg}} \right)
\end{aligned}
\]
Equivalent expressions are obtained to convert activity coefficients expressed in terms of concentration:
\[
\begin{aligned}
\frac{\gamma_i^{IV} c_i/c_i^{\circ}}{\gamma_i^{II} x_i} & = \lim_{x_0 \to 1} \frac{\gamma_i^{IV} c_i/c_i^{\circ}}{\gamma_i^{II} x_i} = \frac{1}{c_i^0} \times \lim_{x_0 \to 1} \frac{c_i}{x_i} = \frac{\rho_0}{c_i^0} \times \lim_{x_0 \to 1} \frac{m_i}{x_i} \\ & = \frac{\rho_0}{c_i^0} \times \lim_{x_0 \to 1}\frac{n_0+\sum_{j \neq 0}n_j}{n_0 M_0} = \frac{\rho_0}{c_i^0 M_0}\\
\end{aligned}
\]
Similarly, the relationship between calculated activity coefficients is determined:
\begin{equation}
\label{eq:act_coef_c_to_x}
\gamma_i^{II} = \gamma_i^{IV} \times \frac{c_i M_0}{x_i \rho_0}
\end{equation}
Equation \ref{eq:act_coef_c_to_x} is represented accordingly with the solute dissociation convention:
\[
\gamma_i^{II} = \gamma_i^{IV} \times \frac{M_0}{\rho_0}\times \frac{c_i}{x_i} = \gamma_i^{IV} \times (1 + \sum_{j\neq0}{m_jM_0}) = \gamma_i^{IV} \times \left(1+\frac{\nu m M_0}{1000 \frac{g}{kg}} \right)
\]
 \begin{thebibliography}{9}
 \bibitem{Denbigh1968} Denbigh, Kenneth G.; The principles of chemical equilibrium; 4th ed. Cambridge University Press, UK 1981.
 \bibitem{Hamer1968} NSRDS 24 Theoretical Mean Activity Coefficients of Strong Electrolytes in Aqueous Solutions from 0 to 100oC - Walter J. Hamer. NSRDS-NBS 24, 271p. (1968). www.nist.gov/data/nsrds/NSRDS-NBS-24.pdf [Apr-2016]
 \end{thebibliography}
\end{document}