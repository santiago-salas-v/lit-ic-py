\documentclass[onecolumn]{article}
\usepackage{amsmath}
\usepackage{lscape}
% wide page for side by side figures, tables, etc
\usepackage{changepage}
\newlength{\offsetpage}
\setlength{\offsetpage}{2.5cm}
\newenvironment{widepage}{\begin{adjustwidth}{-\offsetpage}{-\offsetpage}%
    \addtolength{\textwidth}{2\offsetpage}}%
{\end{adjustwidth}}
\begin{document}
\title{05 - Example 03 - Davies model}
\author{}
\date{}
\maketitle
Example, 5 components (4 + solvent), 3 reactions: \\
\section{Variables}
5 (components + solvent) compositions $n_0$, $n_1$, ..., $n_4$. \\
5 (components + solvent) activity coefficients
$\gamma_0$, $\gamma_1$, ..., $\gamma_4$. \\
3 reaction extents $\xi_1$, $\xi_2$, ..., $\xi_{3}$. \\
1 Ionic strength $I$. \\
Tot. $2(5) + 3 + 1 = 14$ \\
\\
$x =[$
\begin{tabular}{cccccccccccccc}
$n_0$ & $m_1$ & $m_2$ & $m_3$ & $m_4$ & $\xi_1$ & $\xi_2$ & $\xi_3$ &
$\gamma_0$ & $\gamma_1$ & $\gamma_2$ & $\gamma_3$ & $\gamma_4$ & $I$
$]^T$
\end{tabular}
\section{Objective function}
5 mole balances. \\
5 activity coefficient expressions. \\
3 equilibrium expresisons. \\
1 Ionic strength expression. \\
Tot. $2(5) + 3 + 1 = 14$\\
\\
\[
f(x) = 0 = \left(
\begin{tabular}{l}
$-n_0 + n_{0,0} + \nu_{01}\xi_1 + \nu_{02}\xi_2 + \nu_{03}\xi_3  $\\
$-m_1 n_0 M_0 + n_{1,0} + \nu_{11}\xi_1 + \nu_{12}\xi_2 + \nu_{13}\xi_3  $\\
$-m_2 n_0 M_0 + n_{2,0} + \nu_{21}\xi_1 + \nu_{22}\xi_2 + \nu_{23}\xi_3  $\\
$-m_3 n_0 M_0 + n_{3,0} + \nu_{31}\xi_1 + \nu_{32}\xi_2 + \nu_{33}\xi_3  $\\
$-m_4 n_0 M_0 + n_{4,0} + \nu_{41}\xi_1 + \nu_{42}\xi_2 + \nu_{43}\xi_3  $\\
$-K_1 + (\frac{1}{m^{\circ}})^{\sum_j{\nu_{j1}}}\times
\gamma_0^{\nu_{01}}\gamma_1^{\nu_{11}}\gamma_2^{\nu_{21}}
\gamma_3^{\nu_{31}}\gamma_4^{\nu_{41}}\times
\left(\frac{1}{M_0}\right)^{\nu_{01}}m_1^{\nu_{11}}m_2^{\nu_{21}}
m_3^{\nu_{31}}m_4^{\nu_{41}}$\\
$-K_1 + (\frac{1}{m^{\circ}})^{\sum_j{\nu_{j2}}}\times
\gamma_0^{\nu_{02}}\gamma_1^{\nu_{12}}\gamma_2^{\nu_{22}}
\gamma_3^{\nu_{32}}\gamma_4^{\nu_{42}}\times
\left(\frac{1}{M_0}\right)^{\nu_{02}}m_1^{\nu_{12}}m_2^{\nu_{22}}
m_3^{\nu_{32}}m_4^{\nu_{42}}$\\
$-K_1 + (\frac{1}{m^{\circ}})^{\sum_j{\nu_{j3}}}\times
\gamma_0^{\nu_{03}}\gamma_1^{\nu_{13}}\gamma_2^{\nu_{23}}
\gamma_3^{\nu_{33}}\gamma_4^{\nu_{43}}\times
\left(\frac{1}{M_0}\right)^{\nu_{03}}m_1^{\nu_{13}}m_2^{\nu_{23}}
m_3^{\nu_{33}}m_4^{\nu_{43}}$\\
$-\gamma_0 + e^{-1.0\times M_0 \sum_{j\neq0}{m_j}}$ \\
$-\gamma_1 +10^{- 0.510z_1^2 \times
\left(\frac{\sqrt{I}}{1+\sqrt{I}}-0.3I\right) + (1-sign(z_1)^2)b_1 I}$ \\
$-\gamma_2 +10^{- 0.510z_2^2 \times
\left(\frac{\sqrt{I}}{1+\sqrt{I}}-0.3I\right) + (1-sign(z_2)^2)b_2 I}$ \\
$-\gamma_3 +10^{- 0.510z_3^2 \times
\left(\frac{\sqrt{I}}{1+\sqrt{I}}-0.3I\right) + (1-sign(z_3)^2)b_3 I}$ \\
$-\gamma_4 +10^{- 0.510z_4^2 \times
\left(\frac{\sqrt{I}}{1+\sqrt{I}}-0.3I\right) + (1-sign(z_4)^2)b_4 I}$ \\
$-I + \frac{1}{2}(m_0 z_0^2 +  m_1 z_1^2 + m_2 z_2^2 + m_3 z_3^2 + m_4 z_4^2)$
\end{tabular}
\right)
\]
For convenience, define the reaction quotient:\\
\begin{equation}
\label{eq:reaction_quotient}
Q_j \equiv \left(\frac{1}{m^{\circ}}\right)^{\sum_i{\nu_{ij}}}\times
\gamma_0^{\nu_{0j}}\gamma_1^{\nu_{1j}}\gamma_2^{\nu_{2j}}
\gamma_3^{\nu_{3j}}\gamma_4^{\nu_{4j}}\times
\left(\frac{1}{M_0}\right)^{\nu_{0j}}m_1^{\nu_{1j}}m_2^{\nu_{2j}}
m_3^{\nu_{3j}}m_4^{\nu_{4j}}
\end{equation}
Consider non-electrolyte solvent: $z_0 = 0$ \\
Reference: $m^{\circ} = 1\frac{mol}{kg_{solvent}}$
\section{Jacobian}
\[
J(x) = \left(
\begin{tabular}{llll}
$\frac{\partial{f_1}}{\partial{x_1}}$ &
$\frac{\partial{f_1}}{\partial{x_2}}$ & ... &
$\frac{\partial{f_1}}{\partial{x_{14}}}$\\
$\frac{\partial{f_2}}{\partial{x_1}}$ &
$\frac{\partial{f_2}}{\partial{x_2}}$ & ... &
$\frac{\partial{f_2}}{\partial{x_{14}}}$\\
$...$ & $...$ & $...$ & $...$\\
$\frac{\partial{f_{14}}}{\partial{x_1}}$ &
$\frac{\partial{f_{14}}}{\partial{x_2}}$ & ... &
$\frac{\partial{f_{14}}}{\partial{x_{14}}}$\\
\end{tabular}
\right)
\]\\
\[
J_{1,1}(x) = \frac{\partial{f_{1}}}{\partial{x_1}} =  -1
\]
\[
J_{1,2}(x) = \frac{\partial{f_{1}}}{\partial{x_2}} =  0
\]
\[
J_{1,3}(x) = \frac{\partial{f_{1}}}{\partial{x_2}} =  0
\]
(...)
\[
J_{1,6}(x) = \frac{\partial{f_{1}}}{\partial{x_6}} =  \nu_{01}
\]
(...)
\[
J_{2,1}(x) =  -m_1 M_0;
J_{2,2}(x) =  -n_0 M_0;
J_{2,3}(x) =  0;
J_{2,6}(x) = \nu_{12}
\]
(...)
\[
\begin{aligned}
J_{6,1}(x) = & 0 \\
J_{6,2}(x) = &
\left(\frac{1}{m^{\circ}}\right)^{\sum_i{\nu_{i1}}}\times
\gamma_0^{\nu_{01}}\gamma_1^{\nu_{11}}\gamma_2^{\nu_{21}}
\gamma_3^{\nu_{31}}\gamma_4^{\nu_{41}}\times
\left(\frac{1}{M_0}\right)^{\nu_{01}}
\nu_{11}m_1^{\nu_{11}-1}m_2^{\nu_{21}}
m_3^{\nu_{31}}m_4^{\nu_{41}}\\
= & \frac{\nu_{11}}{m_{1}} \times Q_1\\
J_{6,3}(x) =& \frac{\nu_{21}}{m_2} \times Q_1\\
\text{(...)}\\
J_{6,6}(x) =& 0\\
J_{6,7}(x) =& 0\\
J_{6,8}(x) =& 0\\
J_{6,9}(x) =&
\left(\frac{1}{m^{\circ}}\right)^{\sum_i{\nu_{i1}}}\times
\nu_{01}\gamma_0^{\nu_{01}-1}\gamma_1^{\nu_{11}}\gamma_2^{\nu_{21}}
\gamma_3^{\nu_{31}}\gamma_4^{\nu_{41}}\times
\left(\frac{1}{M_0}\right)^{\nu_{01}}m_1^{\nu_{11}}m_2^{\nu_{21}}
m_3^{\nu_{31}}m_4^{\nu_{41}}\\
= & \frac{\nu_{01}}{\gamma_{0}} \times Q_1\\
J_{6,10}(x) =&
\left(\frac{1}{m^{\circ}}\right)^{\sum_i{\nu_{i1}}}\times
\gamma_0^{\nu_{01}}\nu_{11}\gamma_1^{\nu_{11}-1}\gamma_2^{\nu_{21}}
\gamma_3^{\nu_{31}}\gamma_4^{\nu_{41}}\times
\left(\frac{1}{M_0}\right)^{\nu_{01}}m_1^{\nu_{11}}m_2^{\nu_{21}}
m_3^{\nu_{31}}m_4^{\nu_{41}}\\
= & \frac{\nu_{11}}{\gamma_{1}} \times Q_1\\
\text{(...)}\\
J_{6,14}(x) =&  0\\
\text{(...)}\\
\end{aligned}
\]
\[
\begin{aligned}
J_{9,1}(x) =&  0\\
J_{9,2}(x) =& -1.0 M_0 m_1 e^{-1.0\times M_0 \sum_{j\neq0}{m_j}}\\
J_{9,3}(x) =& -1.0 M_0 m_2 e^{-1.0\times M_0 \sum_{j\neq0}{m_j}}\\
\text{(...)}\\
J_{9,5}(x) =& -1.0 M_0 m_4 e^{-1.0\times M_0 \sum_{j\neq0}{m_j}}\\
J_{9,6}(x) =& 0\\
\text{(...)}\\
J_{9,9}(x) =& -1\\
J_{9,10}(x) =& 0\\
J_{9,11}(x) =& 0\\
\text{(...)}\\
J_{9,14}(x) =& 0\\
J_{10,1}(x) =& 0\\
J_{10,2}(x) =& 0\\
\text{(...)}\\
J_{10,9}(x) =& 0\\
J_{10,10}(x) =& -1\\
J_{10,11}(x) =& 0\\
\text{(...)}\\
J_{10,14}(x) =&
-0.510 z_1^2\times\left(\frac{1}{2\sqrt{I}(1+\sqrt{I})} -0.3\right)
\times ln(10) \times 10^{- 0.510z_1^2 \times
\left(\frac{\sqrt{I}}{1+\sqrt{I}}-0.3I\right) + (1-sign(z_1)^2)b_1 I}\\
=& -0.510 z_1^2\times\left(\frac{1}{2\sqrt{I}(1+\sqrt{I})} -0.3\right)
 \times ln(10) \times (\gamma_1+f_{10}(x)) \\
\text{(...)}\\
J_{11,1}(x) =& 0\\
\text{(...)}\\
J_{11,11}(x) =& -1\\
J_{11,12}(x) =& 0\\
\text{(...)}\\
J_{11,14}(x) =&
-0.510 z_2^2\times\left(\frac{1}{2\sqrt{I}(1+\sqrt{I})} -0.3\right)
\times ln(10) \times 10^{- 0.510z_2^2 \times
\left(\frac{\sqrt{I}}{1+\sqrt{I}}-0.3I\right) + (1-sign(z_2)^2)b_2 I}\\
=& -0.510 z_2^2\times\left(\frac{1}{2\sqrt{I}(1+\sqrt{I})} -0.3\right)
 \times ln(10) \times (\gamma_2+f_{11}(x)) \\
\text{(...)}\\
\end{aligned}
\]
Using the reaction quotients $Q_i$ for convenience as defined above (eq.
\ref{eq:reaction_quotient}), the Jacobian matrix results as follows,
considering non-electrolyte solvent (charge $z_0=0$):
\begin{landscape}
\[
J(x) = \left(
\begin{tabular}{llllllllllllll}
$-1$ & 0 & 0 & 0 & 0 &
$\nu_{01}$ & $\nu_{02}$ & $\nu_{03}$ &
0 & 0 & 0 & 0 & 0 &
0\\
0 & $-m_1 M_0$ & 0 & 0 & 0 &
$\nu_{11}$ & $\nu_{12}$ & $\nu_{13}$ &
0 & 0 & 0 & 0 & 0 &
0\\
0 & 0 & $-m_2 M_0$  & 0 & 0 &
$\nu_{21}$ & $\nu_{22}$ & $\nu_{23}$ &
0 & 0 & 0 & 0 & 0 &
0\\
0 & 0 & 0 & $-m_3 M_0$ & 0 &
$\nu_{31}$ & $\nu_{32}$ & $\nu_{33}$ &
0 & 0 & 0 & 0 & 0 &
0\\
0 & 0 & 0 & 0 & $-m_4 M_0$ &
$\nu_{41}$ & $\nu_{42}$ & $\nu_{43}$ &
0 & 0 & 0 & 0 & 0 &
0\\
0 &
$\frac{\nu_{11}}{m_1}Q_1$ & $\frac{\nu_{21}}{m_2}Q_1$ &
$\frac{\nu_{31}}{m_3}Q_1$ & $\frac{\nu_{41}}{m_4}Q_1$ &
0 & 0 & 0 &
$\frac{\nu_{01}}{\gamma_0}Q_1$ & $\frac{\nu_{11}}{\gamma_1}Q_1$ &
$\frac{\nu_{21}}{\gamma_2}Q_1$ & $\frac{\nu_{31}}{\gamma_3}Q_1$ &
$\frac{\nu_{41}}{\gamma_4}Q_1$ &
0\\
0 &
$\frac{\nu_{12}}{m_1}Q_2$ & $\frac{\nu_{22}}{m_2}Q_2$ &
$\frac{\nu_{32}}{m_3}Q_2$ & $\frac{\nu_{42}}{m_4}Q_2$ &
0 & 0 & 0 &
$\frac{\nu_{02}}{\gamma_0}Q_2$ & $\frac{\nu_{12}}{\gamma_1}Q_2$ &
$\frac{\nu_{22}}{\gamma_2}Q_2$ & $\frac{\nu_{32}}{\gamma_3}Q_2$ &
$\frac{\nu_{42}}{\gamma_4}Q_2$ &
0\\
0 &
$\frac{\nu_{13}}{m_1}Q_3$ & $\frac{\nu_{23}}{m_2}Q_3$ &
$\frac{\nu_{33}}{m_3}Q_3$ & $\frac{\nu_{43}}{m_4}Q_3$ &
0 & 0 & 0 &
$\frac{\nu_{03}}{\gamma_0}Q_3$ & $\frac{\nu_{13}}{\gamma_1}Q_3$ &
$\frac{\nu_{23}}{\gamma_2}Q_3$ & $\frac{\nu_{33}}{\gamma_3}Q_3$ &
$\frac{\nu_{43}}{\gamma_4}Q_3$ &
0\\
0 & $J_{9,2}(x)$ & $J_{9,3}(x)$ & $J_{9,4}(x)$ & $J_{9,5}(x)$ &
0 & 0 & 0 &
-1 & 0 & 0 & 0 & 0 &
0\\
0 & 0 & 0 & 0 & 0 &
0 & 0 & 0 &
0 & -1 & 0 & 0 & 0 &
$J_{10,14}(x)$\\
0 & 0 & 0 & 0 & 0 &
0 & 0 & 0 &
0 & 0 & -1 & 0 & 0 &
$J_{11,14}(x)$\\
0 & 0 & 0 & 0 & 0 &
0 & 0 & 0 &
0 & 0 & 0 & -1 & 0 &
$J_{12,14}(x)$\\
0 & 0 & 0 & 0 & 0 &
0 & 0 & 0 &
0 & 0 & 0 & 0 & -1 &
$J_{13,14}$\\
0 &
$\frac{1}{2}z_1^2$ & $\frac{1}{2}z_2^2$ &
$\frac{1}{2}z_3^2$ & $\frac{1}{2}z_4^2$ &
0 & 0 & 0 &
0 & 0 & 0 & 0 & 0 &
-1\\
\end{tabular}
\right)
\]
\[
\begin{aligned}
J_{9,2}(x) =& -1.0 M_0 m_1 e^{-1.0\times M_0 \sum_{j\neq0}{m_j}}\\
J_{9,3}(x) =& -1.0 M_0 m_2 e^{-1.0\times M_0 \sum_{j\neq0}{m_j}}\\
J_{9,4}(x) =& -1.0 M_0 m_3 e^{-1.0\times M_0 \sum_{j\neq0}{m_j}}\\
J_{9,5}(x) =& -1.0 M_0 m_4 e^{-1.0\times M_0 \sum_{j\neq0}{m_j}}\\
J_{10,14}(x) =&
-0.510 ln(10)\times z_1^2\times\left(\frac{1}{2\sqrt{I}(1+\sqrt{I})} -0.3\right)
\times 10^{- 0.510z_1^2 \times
\left(\frac{\sqrt{I}}{1+\sqrt{I}}-0.3I\right) + (1-sign(z_1)^2)b_1 I}\\
J_{11,14}(x) =&
-0.510 ln(10)\times z_2^2\times\left(\frac{1}{2\sqrt{I}(1+\sqrt{I})} -0.3\right)
\times 10^{- 0.510z_2^2 \times
\left(\frac{\sqrt{I}}{1+\sqrt{I}}-0.3I\right) + (1-sign(z_2)^2)b_2 I}\\
J_{12,14}(x) =&
-0.510 ln(10)\times z_3^2\times\left(\frac{1}{2\sqrt{I}(1+\sqrt{I})} -0.3\right)
\times 10^{- 0.510z_3^2 \times
\left(\frac{\sqrt{I}}{1+\sqrt{I}}-0.3I\right) + (1-sign(z_3)^2)b_3 I}\\
J_{13,14}(x) =&
-0.510 ln(10)\times z_4^2\times\left(\frac{1}{2\sqrt{I}(1+\sqrt{I})} -0.3\right)
\times 10^{- 0.510z_4^2 \times
\left(\frac{\sqrt{I}}{1+\sqrt{I}}-0.3I\right) + (1-sign(z_4)^2)b_4 I}\\
\end{aligned}
\]
\end{landscape}
For calculation, submatrixes of the Jacobian matrix are identified:
\begin{itemize}
\item $J_{1,1}(x)$ to $J_{n+1,n+1}(x)$ is a diagonal matrix of $n_i/n_0$
as function of molality:
\[
\begin{aligned}
J_{1,1}(x) \text{ to } J_{n+1,n+1}(x) = &
-diag( \left[
\begin{tabular}{ccccc}
1 & $m_1 M_0$ & $m_2 M_0$ & $m_3 M_0$ & $m_4 M_0$\\
\end{tabular}
\right]
)\\
= & \left(
\begin{tabular}{ccccc}
-1 & 0 & 0 & 0 & 0\\
0 & -$m_1 M_0$ & 0 & 0 & 0\\
0 & 0 & -$m_2 M_0$ & 0 & 0\\
0 & 0 & 0 & -$m_3 M_0$ & 0\\
0 & 0 & 0 & 0 & -$m_4 M_0$\\
\end{tabular}
\right)
\end{aligned}
\]
\item $J_{n+2+nr,n+2+nr}(x)$ to $J_{n+2+nr+n+1,n+2+nr+n+1}(x)$
is the negative of an identity matrix, size $(n+1) \times (n+1)$
\[
J_{n+2+nr,n+2+nr}(x) \text{ to } J_{n+2+nr+n+1,n+2+nr+n+1}(x) =
-I_{n+1} = -1\times \left(
\begin{tabular}{ccccc}
1 & 0 & 0 & 0 & 0\\
0 & 1 & 0 & 0 & 0\\
0 & 0 & 1 & 0 & 0\\
0 & 0 & 0 & 1 & 0\\
0 & 0 & 0 & 0 & 1\\
\end{tabular}
\right)
\]
\item $J_{1,n+2}(x)$ to $J_{n+1,n+2+nr}(x)$ is the stoichiometric coefficients
matrix
\[
J_{1,n+2}(x) \text{ to } J_{n+1,n+2+nr}(x) = \nu_{ij} =\left(
\begin{tabular}{ccc}
$\nu_{01}$ & $\nu_{02}$ & $\nu_{03}$\\
$\nu_{11}$ & $\nu_{12}$ & $\nu_{13}$\\
$\nu_{21}$ & $\nu_{22}$ & $\nu_{23}$\\
$\nu_{31}$ & $\nu_{32}$ & $\nu_{33}$\\
$\nu_{41}$ & $\nu_{42}$ & $\nu_{43}$\\
\end{tabular}
\right)
\]
\item $J_{n+2,1}(x)$ to $J_{n+2+nr,n+1}(x)$ is expressed as a product of the
diagonalized quotients matrix, the transposed coefficients matrix,
and a diagonalized inverse molality matrix, with no variation resulting from
solvent:
\[
\begin{aligned}
J_{n+2,1}(x) & \text{ to } J_{n+2+nr,n+1}(x) \\
= & diag([Q_1 Q_2 Q_3])\times \nu_{ij}^T \times
diag([0 \frac{1}{m_1} \frac{1}{m_2} \frac{1}{m_3} \frac{1}{m_4}]) \\
= & \left(
\begin{tabular}{ccc}
$Q_1$ & 0 & 0\\
0 & $Q_2$ & 0\\
0 & 0 & $Q_3$\\
\end{tabular}
\right) \times \left(
\begin{tabular}{ccccc}
$\nu_{01}$ & $\nu_{11}$ & $\nu_{21}$ & $\nu_{31}$ & $\nu_{41}$\\
$\nu_{02}$ & $\nu_{12}$ & $\nu_{22}$ & $\nu_{32}$ & $\nu_{42}$\\
$\nu_{03}$ & $\nu_{13}$ & $\nu_{23}$ & $\nu_{33}$ & $\nu_{43}$\\
\end{tabular}
\right) \times \left(
\begin{tabular}{ccccc}
0 & 0 & 0 & 0 & 0\\
0 & $\frac{1}{m_1}$ & 0 & 0 & 0\\
0 & 0 & $\frac{1}{m_2}$ & 0 & 0\\
0 & 0 & 0 & $\frac{1}{m_3}$ & 0\\
0 & 0 & 0 & 0 & $\frac{1}{m_4}$\\
\end{tabular}
\right)\\
= & \left(
\begin{tabular}{ccccc}
0 &
$\frac{\nu_{11}}{m_1}Q_1$ & $\frac{\nu_{21}}{m_2}Q_1$ &
$\frac{\nu_{31}}{m_3}Q_1$ & $\frac{\nu_{41}}{m_4}Q_1$\\
0 &
$\frac{\nu_{12}}{m_1}Q_2$ & $\frac{\nu_{22}}{m_2}Q_2$ &
$\frac{\nu_{32}}{m_3}Q_2$ & $\frac{\nu_{42}}{m_4}Q_2$\\
0 &
$\frac{\nu_{13}}{m_1}Q_3$ & $\frac{\nu_{23}}{m_2}Q_3$ &
$\frac{\nu_{33}}{m_3}Q_3$ & $\frac{\nu_{43}}{m_4}Q_3$\\
\end{tabular}
\right)
\end{aligned}
\]
\item $J_{n+2,n+2+nr}(x)$ to $J_{n+2+nr,n+2+nr+n+1}(x)$ is analogous for
activity coefficient partial variations:
\[
\begin{aligned}
J_{n+2,n+2+nr}(x) & \text{ to } J_{n+2+nr,n+2+nr+n+1}(x) \\
= & diag([Q_1 Q_2 Q_3])\times \nu_{ij}^T \times
diag([\frac{1}{\gamma_0} \frac{1}{\gamma_1} \frac{1}{m_2}
\frac{1}{m_3} \frac{1}{m_4}]) \\
= & \left(
\begin{tabular}{ccc}
$Q_1$ & 0 & 0\\
0 & $Q_2$ & 0\\
0 & 0 & $Q_3$\\
\end{tabular}
\right) \times  \\
& \left(
\begin{tabular}{ccccc}
$\nu_{01}$ & $\nu_{11}$ & $\nu_{21}$ & $\nu_{31}$ & $\nu_{41}$\\
$\nu_{02}$ & $\nu_{12}$ & $\nu_{22}$ & $\nu_{32}$ & $\nu_{42}$\\
$\nu_{03}$ & $\nu_{13}$ & $\nu_{23}$ & $\nu_{33}$ & $\nu_{43}$\\
\end{tabular}
\right) \times
\left(
\begin{tabular}{ccccc}
$\frac{1}{\gamma_0}$ & 0 & 0 & 0 & 0\\
0 & $\frac{1}{\gamma_1}$ & 0 & 0 & 0\\
0 & 0 & $\frac{1}{\gamma_2}$ & 0 & 0\\
0 & 0 & 0 & $\frac{1}{\gamma_3}$ & 0\\
0 & 0 & 0 & 0 & $\frac{1}{\gamma_4}$\\
\end{tabular}
\right)\\
= & \left(
\begin{tabular}{ccccc}
$\frac{\nu_{01}}{\gamma_0}Q_1$ & $\frac{\nu_{11}}{\gamma_1}Q_1$ &
$\frac{\nu_{21}}{\gamma_2}Q_1$ & $\frac{\nu_{31}}{\gamma_3}Q_1$ &
$\frac{\nu_{41}}{\gamma_4}Q_1$\\
$\frac{\nu_{02}}{\gamma_0}Q_2$ & $\frac{\nu_{12}}{\gamma_1}Q_2$ &
$\frac{\nu_{22}}{\gamma_2}Q_2$ & $\frac{\nu_{32}}{\gamma_3}Q_2$ &
$\frac{\nu_{42}}{\gamma_4}Q_2$\\
$\frac{\nu_{03}}{\gamma_0}Q_3$ & $\frac{\nu_{13}}{\gamma_1}Q_3$ &
$\frac{\nu_{23}}{\gamma_2}Q_3$ & $\frac{\nu_{33}}{\gamma_3}Q_3$ &
$\frac{\nu_{43}}{\gamma_4}Q_3$\\
\end{tabular}
\right)
\end{aligned}
\]
\end{itemize}
\section{Solution}
Use Newton-Rhapson method with objective function $f(x)=0$, Jacobian $J(x)$,
with LR=PDA factorization for optimal jacobian condition and line search
(batcracking) to remain on valid steps fulfilling all $m_i>0$
\end{document}
